\section{Introduction}

\subsection{Foreward}

My original plan was to publish and sell Condensed Trigonometry in mid 2014.  Unfortunately, I haven't had the time to deal with traditional publishing.  So, in an effort to have the book make some impact, I've decided to release it under the Creative Commons license.  Under this license, others may read, modify, and redistribute the full text and LaTeX source code it as they see fit.\\

If you look today, you'll often find trigonometry textbooks that cost \$100 or more.  That's quite a premium for a 2000 year old idea!  I hope you find this to be an acceptable alternative, and if not, let me know!  Constructive criticism is welcome, and I'll keep working to improve the text and regularly release updates.\\

\subsection{License}
This work is licensed under the Creative Commons Attribution 4.0 International License. To view a copy of this license, visit \\ http://creativecommons.org/licenses/by/4.0.

\subsection{Instructions}

This book is an efficient introduction to trigonometry   As a result, ideas and definitions are presented rapidly.  Stop as often as you need to reflect on a sentence or equation.  Be sure you understand and memorize the contents of {\bf every} page.  If you don't understand a term or a symbol, stop and look it up before continuing to read.  Re-read a chapter as often as you need to understand it.\\

At the end of every section there is a set of review questions.  These questions are designed to test your understanding of the concepts presented in the section.  Be sure you answer {\bf every} question before moving on.  You should be able to solve every problem without looking at the book.  The solution for every question can be found in the back of the book.  Re-solve every problem as many times as you need.\\

If you can recall the contents of every chapter, and can solve every review problem without difficulty, you can say with confidence that you understand trigonometry.\\

\clearpage
\subsection{Table of Symbols}

\begin{figure}[htb]
\caption{Table of symbols used in the book.}
\label{fig:table_of_symbols}
\begin{center}
\begin{tabular}{ |c| l |}
\hline 
$\theta$ & The Greek letter theta; used as the measure of an angle.\\
\hline 
$l$ & The length of an arc.\\
\hline 
$r$ & The radius of an arc.\\
\hline 
$\approx$ & 'Approximately equal'.\\
\hline 
$\pi$ & The Greek letter pi;  a special constant $\approx 3.14$ .\\
\hline
$rad$ & Shorthand notation for radians.\\
\hline
$^c$ & Shorthand notation for radians.\\
\hline
$deg$ & Shorthand notation for degrees.\\
\hline
$^o$ & Shorthand notation for degrees.\\
\hline
$nan$ & Shorthand for 'not a number'.\\
\hline
$\implies$ & Shorthand for 'implies'.  Used in derivations.\\
\hline
$i$ & An imaginary number.  Shorthand for $sqrt(-1)$\\
\hline
$e$ & Euler's number; a special constant $\approx 2.78$\\
\hline
\end{tabular}
\end{center}
\end{figure}

